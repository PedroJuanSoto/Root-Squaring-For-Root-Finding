%%
%% This is file `sample-sigconf.tex',
%% generated with the docstrip utility.
%%
%% The original source files were:
%%
%% samples.dtx  (with options: `sigconf')
%%
%% IMPORTANT NOTICE:
%%
%% For the copyright see the source file.
%%
%% Any modified versions of this file must be renamed
%% with new filenames distinct from sample-sigconf.tex.
%%
%% For distribution of the original source see the terms
%% for copying and modification in the file samples.dtx.
%%
%% This generated file may be distributed as long as the
%% original source files, as listed above, are part of the
%% same distribution. (The sources need not necessarily be
%% in the same archive or directory.)
%%
%%
%% Commands for TeXCount
%TC:macro \cite [option:text,text]
%TC:macro \citep [option:text,text]
%TC:macro \citet [option:text,text]
%TC:envir table 0 1
%TC:envir table* 0 1
%TC:envir tabular [ignore] word
%TC:envir displaymath 0 word
%TC:envir math 0 word
%TC:envir comment 0 0
%%
%%
%% The first command in your LaTeX source must be the \documentclass command.
\documentclass[sigconf]{acmart}
\usepackage{amsmath}
\usepackage{algorithm}
\usepackage{algorithmic}
\usepackage{amsthm}

%%
%% \BibTeX command to typeset BibTeX logo in the docs
\AtBeginDocument{%
  \providecommand\BibTeX{{%
    \normalfont B\kern-0.5em{\scshape i\kern-0.25em b}\kern-0.8em\TeX}}}

%% Rights management information.  This information is sent to you
%% when you complete the rights form.  These commands have SAMPLE
%% values in them; it is your responsibility as an author to replace
%% the commands and values with those provided to you when you
%% complete the rights form.
\setcopyright{acmcopyright}
\copyrightyear{2022}
\acmYear{2022}
\acmDOI{XXXXXXX.XXXXXXX}

%% These commands are for a PROCEEDINGS abstract or paper.
\acmConference[ISSAC '22]{International Symposium on
Symbolic and Algebraic Computation}{July 04--07,
  2022}{Lille, France}
% \acmPrice{15.00}
% \acmISBN{978-1-4503-XXXX-X/18/06}


%%
%% Submission ID.
%% Use this when submitting an article to a sponsored event. You'll
%% receive a unique submission ID from the organizers
%% of the event, and this ID should be used as the parameter to this command.
%%\acmSubmissionID{123-A56-BU3}

%%
%% The majority of ACM publications use numbered citations and
%% references.  The command \citestyle{authoryear} switches to the
%% "author year" style.
%%
%% If you are preparing content for an event
%% sponsored by ACM SIGGRAPH, you must use the "author year" style of
%% citations and references.
%% Uncommenting
%% the next command will enable that style.
%%\citestyle{acmauthoryear}

%%
%% end of the preamble, start of the body of the document source.
\begin{document}

%%
%% The "title" command has an optional parameter,
%% allowing the author to define a "short title" to be used in page headers.
\title{Root-Squaring for Root-Finding}

%%
%% The "author" command and its associated commands are used to define
%% the authors and their affiliations.
%% Of note is the shared affiliation of the first two authors, and the
%% "authornote" and "authornotemark" commands
%% used to denote shared contribution to the research.
\author{Pedro Soto}
\affiliation{%
    \institution{The Graduate Center, CUNY}
    \city{New York}
    \state{New York}
    \country{USA}
}
\email{psoto@gradcenter.cuny.edu}



\author{Soo Go}
\affiliation{%
    \institution{The Graduate Center, CUNY}
    \city{New York}
    \state{New York}
    \country{USA}
}
\email{sgo@gradcenter.cuny.edu}
%
% \author{Victor}
% \affiliation{%
%     \institution{CUNY}
%     \city{NY}
%     \state{NY}
%     \country{USA}
% }

% \author{Aparna Patel}
% \affiliation{%
%  \institution{Rajiv Gandhi University}
%  \streetaddress{Rono-Hills}
%  \city{Doimukh}
%  \state{Arunachal Pradesh}
%  \country{India}}
%
% \author{Huifen Chan}
% \affiliation{%
%   \institution{Tsinghua University}
%   \streetaddress{30 Shuangqing Rd}
%   \city{Haidian Qu}
%   \state{Beijing Shi}
%   \country{China}}
%
% \author{Charles Palmer}
% \affiliation{%
%   \institution{Palmer Research Laboratories}
%   \streetaddress{8600 Datapoint Drive}
%   \city{San Antonio}
%   \state{Texas}
%   \country{USA}
%   \postcode{78229}}
% \email{cpalmer@prl.com}
%
% \author{John Smith}
% \affiliation{%
%   \institution{The Th{\o}rv{\"a}ld Group}
%   \streetaddress{1 Th{\o}rv{\"a}ld Circle}
%   \city{Hekla}
%   \country{Iceland}}
% \email{jsmith@affiliation.org}
%
% \author{Julius P. Kumquat}
% \affiliation{%
%   \institution{The Kumquat Consortium}
%   \city{New York}
%   \country{USA}}
% \email{jpkumquat@consortium.net}

%%
%% By default, the full list of authors will be used in the page
%% headers. Often, this list is too long, and will overlap
%% other information printed in the page headers. This command allows
%% the author to define a more concise list
%% of authors' names for this purpose.
\renewcommand{\shortauthors}{Soto and Go, et al.}

%%
%% The abstract is a short summary of the work to be presented in the
%% article.
\begin{abstract}
We revisited the classical root-squaring formula of Dandelin-Lobachevsky-Graeffe for polynomials and found new interesting applications to root-finding.
\end{abstract}

%%
%% The code below is generated by the tool at http://dl.acm.org/ccs.cfm.
%% Please copy and paste the code instead of the example below.
%%
\begin{CCSXML}
<ccs2012>
   <concept>
       <concept_id>10010147.10010148.10010149.10010154</concept_id>
       <concept_desc>Computing methodologies~Hybrid symbolic-numeric methods</concept_desc>
       <concept_significance>500</concept_significance>
       </concept>
 </ccs2012>
\end{CCSXML}

\ccsdesc[500]{Computing methodologies~Hybrid symbolic-numeric methods}



%%
%% Keywords. The author(s) should pick words that accurately describe
%% the work being presented. Separate the keywords with commas.
\keywords{symbolic-numeric computing, root finding, polynomial algorithms, computer algebra}

%% A "teaser" image appears between the author and affiliation
%% information and the body of the document, and typically spans the
%% page.
% \begin{teaserfigure}
%   \includegraphics[width=\textwidth]{sampleteaser}
%   \caption{Seattle Mariners at Spring Training, 2010.}
%   \Description{Enjoying the baseball game from the third-base
%   seats. Ichiro Suzuki preparing to bat.}
%   \label{fig:teaser}
% \end{teaserfigure}

%%
%% This command processes the author and affiliation and title
%% information and builds the first part of the formatted document.
\maketitle

\section{Introduction}
 We revisit the famous methods simaltanously discovered by Dandelin, Lobachevsky, and Graeffe (See \cite{10.2307/2310626} for a history of this problem) and and make furhter progress by applying this indentity to the problem of approximating the root radius of a polynomial.

\section{Related Works}

\section{Background}\label{srrapp}
% %------------------------------------------------------------------------------
% \subsection{Root radius and root-squaring}\label{srrapp}
%
% %------------------------------------------------------------------------------.
% The algorithm of Sec. \ref{scnttstB1} and
%  \ref{scnttstB} only ensures
%  approximation of a root radius
% within  factors  of  $d^{1.25}$ and
% $d^{4.75}/m$, respectively (see   Algs.  \ref{algscrincl}$m$ and \ref{algscrcmpr}), versus  a factor of $5^{1/N}$ for, say, $N=3$  ensured by Thm. \ref{thtrn}, and this  is translated into the same error factors for estimated rigidity of the output disc.
% The  difference little affects our estimates for the overall cost of subdivision root-finders (see part (c) of Remark
% \ref{recmprlg}), but not so in some other important applications, e.g., in the following two cases:
% \begin{itemize}
% \item
% A disc $D(c,\rho)$ contains  a root if and only if
% $r_d(c,p)\le \rho$.
% \item
%  The estimates for $r_d(c,p')$  are
%   involved  in  path-lifting Newton's iterations (see  Myong-Hi Kim
%   and Scott Sutherland  \cite{KS94}).
% \end{itemize}
%
% %We also uses a disc $D(c,\alpha r_1(c,p))$ for
% %%\alpha\ge 1$ to initialize  subdivision
% %and Newton's iterations for root-finding.
% This motivates the following problem:
% given
% a range $[\rho_-,\rho_+]$ for  the $m$th smallest root radius, $r_{d-m+1}$, narrow this range.
% We already addressed this problem
%  in Remark \ref{retrhob}, but next we cover alternative  refinement of root radii  approximation, based on the classical technique of recursive root-squaring, which also enables us to relax restriction of Rule 2 of Sec. \ref{ssbdbrf} on softness of exclusion test and to increase isolation of a disc (see, e.g.,  Remark \ref{resmplini}).
%
%  First make an input polynomial $p(x)$ monic by scaling it and/or the variable
%   $x$ and then
%  apply $k$ DLG (that is, Dandelin's aka
% Lobachevsky's or Gr{\"a}ffe's) root-squaring iterations
% (cf. \cite{H59}),
% \begin{equation}\label{eqdnd}
%  p_0(x)=\frac{1}{p_d}p(x),~p_{i+1}(x)=(-1)^ dp_i(\sqrt x)
% p_i(-\sqrt x),~i=0,1,\dots.\ell
% \end{equation}
% for a fixed positive integer $\ell$
% (see Remark \ref{rescdnd} below).
% The $i$th  iteration squares the roots of $p_i(x)$ and
% consequently the root radii from the origin, as well as
% the  isolation of the unit disc $D(0,1)$.  Now approximate
% the ratio  $\rho_+/\rho_-$ for the polynomial $p_{\ell}(x)$ within a factor of $\gamma$ and then readily recover approximation of this ratio for $p_0(x)$ and
% $p(x)$ within a factor of
% $\gamma^{1/2^{\ell}}$.
%
% Given the coefficients of
% $p_i(x)$ , e.g., a deflated factor of $p$,  we can reduce the $i$th root-squaring iteration, that is, the computation of
% the coefficients of  $p_{i+1}(x)$, to polynomial multiplication and perform it in $O(d\log(d))$ arithmetic operations.
% Unless the positive integer $\ell$ is small, the absolute values of the coefficients
% of $p_{\ell}(x)$ vary  dramatically, and realistically one should either stop because of severe problems of numerical stability or apply the stable algorithm by Gregorio Malajovich and Jorge  P. Zubelli \cite{MZ01}, which performs a single root-squaring at  arithmetic cost of order $d^2$.
%
% For black box polynomials $p$, however, we apply
% DLG iterations without computing
% the coefficients, and the algorithm turns out to be quite efficient:
%  for $\ell$ iterations  evaluate
%  $p(x)$ at  $2^{\ell}$ equally spaced points  on a circle and obtain
%  the values of the polynomial $p_{\ell}(x)=\prod(x-x_j^{2^{\ell}})$
%  at these $2^{\ell}$ points.
%
% Furthermore,  evaluate the ratio
% $p'(x)/p(x)=p_0'(x)/p_0(x)$
% at these points by applying
% the recurrence
%  \begin{equation}\label{eqdndrt} \frac{p_{i+1}'(x)}{p_{i+1}(x)}=\frac{1}{2\sqrt x}\Big(\frac{p_{i}'(\sqrt x~)}{p_{i}(\sqrt x~)}-\frac{p_{i}'(-\sqrt x~)}{p_{i}(-\sqrt x~)}\Big),~i=0,1,\dots
% \end{equation}
% Recurrences (\ref{eqdnd}) and (\ref{eqdndrt}) reduce evaluation  of  $p_{\ell}(c)$ to the evaluation of $p(c)$ at  $q=2^{\ell}$  points
% $c^{1/q}$ and  for $c\neq 0$ reduce evaluation  of
%  the ratio
% $p_{\ell}'(x)/p_{\ell}(x)$ at $x=c$ to the evaluation
% of the ratio
% $p'(x)/p(x)$ at the latter $q=2^{\ell}$ points $x=c^{1/q}$.
% We apply  recurrence (\ref{eqdndrt})
% to support application of  Cor. \ref{coqudr} (see Remark \ref{resmplini}).
%
% For $x=0$ recurrence (\ref{eqdndrt}) can be specialized as follows:
%  \begin{equation}\label{eqdndrt0} \frac{p_{{\ell}}'(0)}{p_{\ell}(0)}=\Big(\frac{p_{\ell-1}'(x)}{p_{\ell-1}(x)}\Big)_{x=0}'=\Big(\frac{p'(x)}{p(x)}\Big)_{x=0}^{(\ell)},~\ell=1,2,\dots
% \end{equation}
% Notice an immediate extension:
% \begin{equation}\label{eqdndrth} \frac{p_{{\ell}}^{(h)}(0)}{p_{\ell}(0)}=\prod_{g=1}^h\Big(\frac{p^{(g)}(x)}{p^{(g-1)}(x)}\Big)_{x=0}^{(\ell)},~h=1,2,\dots.
% \end{equation}
% Equations (\ref{eqdndrt0}) and more generally (\ref{eqdndrth})
% enable us to  strengthen
% upper estimates  (\ref{eqrtrdbndsrev1})
% and more generally (\ref{eqrtrdbnds-+i}) for  root-radii  $r_j(0,p)$   at the origin   because $r_j(0,p_{\ell})=r_j(0,p)^{2^{\ell}}$ for $j=1,\dots,d$ (see  (\ref{eqratio0})); we can  approximate the higher order derivatives
% $\Big(\frac{p^{(g)}(x)}{p^{(g-1)}(x)}\Big)^{(\ell)}$
%  at $x=0$ by following Remark \ref{rehighr}.
% Besides listed applications of
% root-squaring, we recall one in Sec. \ref{scorrcch}, for decreasing the softness  of Alg. \ref{algexclrndspr}.
% One can apply root-squaring $p(x)\rightarrow p_{\ell}(x)$ to improve  the error  bound of Cor. \ref{copwrsm} for the approximation of   the power sums  of the roots of  $p(x)$ in the unit disc $D(0,1)$ by Cauchy sums, but the improvement is about as much and at the same additional cost as by increasing the number $q$ of points of evaluation of the ratio $p'/p$.
% \begin{remark}\label{rescdnd}
%  One can approximate the leading coefficient $p_d$  of a black box polynomial $p(x)$ based on equation (\ref{eqpolyrevlc}). This coefficient is not involved in recurrence (\ref{eqdndrt}),
%  and one can apply  recurrence (\ref{eqdnd}) by using a crude
%  approximation to  $p_d$
%  and if needed can scale polynomials
%  $p_i(x)$ for some $i$.
% \end{remark}

\section{Motivating Example}

\begin{figure}[h]
  \centering
  \includegraphics[width=\linewidth]{rational_root_tree.png}
  \caption{The upper tree depicts the steps of \textsc{\textsc{Circle\_Roots\_Rational\_Form}}($p,q,l$) in Alg.\ref{alg:circ_roots_rational_form} for $l=2$, $p=1$, and $q=1$. The lower tree depicts the steps of \textsc{Roots}($r,t,u,l$) in Alg.\ref{alg:roots} for $r=2$, $l=2$, $p=1$, and $q=1$}
  \Description{}
\end{figure}

\begin{figure}[h]
  \centering
  \includegraphics[width=\linewidth]{p_prime.png}
  \caption{The steps of \textsc{DLG\_Rational\_Rorm}($p,p^\prime,r,t,u,l$) in Alg.\ref{alg:DLG_rational_form} for $r=2$, $l=2$, $t=1$, and $u=1$.}
  \Description{}
\end{figure}



\section{Algorithm Design}


\begin{algorithm}
   \caption{\textsc{Circle\_Roots\_Rational\_Form}($p,q,l$)}
   \label{alg:circ_roots_rational_form}
\begin{algorithmic}
\IF{ $p\%q$ == 0}
  \STATE  $r, s$ := (1,1)
\ELSE
  \STATE  $r, s$ := ($p$,$2q$)
\ENDIF
  \IF{ r\%s == 0}
    \STATE $t, u$ := (1,2)
  \ELSE
    \STATE $t, u$ := $(2r+s, 2s)$
  \ENDIF
	  \IF{$l$ == 1}
		  \RETURN [($r,s$),($t,u$)]
	\ELSIF {$l$ != 0}
		\STATE left  := \textsc{Circle\_Roots\_Rational\_Form}($r,s,l-1$)
		\STATE right := \textsc{Circle\_Roots\_Rational\_Form}($t,u,l-1$)
		\RETURN left $\cup$ right
	\ELSE
		\RETURN  [($p,q$)]
      \ENDIF
\end{algorithmic}
\end{algorithm}

\begin{algorithm}
\caption{\textsc{Roots}($r,t,u,l$)}
\label{alg:roots}
\begin{algorithmic}
\STATE root\_tree = \textsc{Circle\_Roots\_Rational\_Form}($p,q,l$)
\STATE circ\_root = [$\exp\left(2\cdot\pi\cdot i \cdot \frac{r}{s}\right)$ for $r,s$ in root\_tree]
% \STATE circ\_root       = circ\_roots($t,u,l$)
\STATE roots =[$\sqrt[2^l]{r}\cdot$root for root in circ\_root]
\RETURN roots
\end{algorithmic}
\end{algorithm}


% \begin{algorithm}
%    \caption{circ\_roots\_rational\_form($p,q,l$)}
%    \label{alg:circ_roots_rational_form}
% \begin{algorithmic}
%   \STATE $r, s$  := angle\_sq\_root($p,q$)
% 	\STATE $t, u$  := angle\_neg($r,s$)
% 	  \IF{$l$ == 1}
% 		  \RETURN [($r,s$),($t,u$)]
% 	\ELSIF {$l$ != 0}
% 		\STATE left  := circ\_roots\_rational\_form($r,s,l-1$)
% 		\STATE right := circ\_roots\_rational\_form($t,u,l-1$)
% 		\RETURN left $\cup$ right
% 	\ELSE
% 		\RETURN  [($p,q$)]
%       \ENDIF
% \end{algorithmic}
% \end{algorithm}


% \begin{algorithm}
% \caption{angle\_sq\_root(p,q)}
% \label{alg:angle_sq_root}
% \begin{algorithmic}
% 	\IF{ $p\%q$ == 0}
% 		\RETURN  (1,1)
% 	\ELSE
% 		\RETURN  ($p$,$2q$)
%   \ENDIF
% \end{algorithmic}
% \end{algorithm}
%
%
% \begin{algorithm}
% \caption{angle\_neg(p,q)}
% \label{alg:angle_neg}
% \begin{algorithmic}
% 	\IF{ p\%q == 0}
% 		\RETURN  (1,2)
% 	\ELSE
% 		\RETURN  $(2p+q, 2q)$
%   \ENDIF
% \end{algorithmic}
% \end{algorithm}




% \begin{algorithm}
% \caption{circ\_roots($p,q,l$):}
% \label{alg:circ_roots}
% \begin{algorithmic}
% \STATE roots = circ\_roots\_rational\_form($p,q,l$)
% \RETURN [$\exp\left(2\cdot\pi\cdot i \cdot \frac{r}{s}\right)$ for r,s in roots]
% \end{algorithmic}
% \end{algorithm}





% \clearpage

\begin{algorithm}
\caption{\textsc{DLG\_Rational\_Form}($p,p^\prime,r,t,u,l$)}
\label{alg:DLG_rational_form}
\begin{algorithmic}
\STATE 	root      := \textsc{Roots}($r,t,u,l$)
\FOR {$r_i \in $ root}
\STATE 	base\_step[$i$] := $\frac{p^\prime(r_i)}{p(r_i)}$
\ENDFOR
\STATE  diff[0]   := base\_step
\FOR {$i \leq l$}
\FOR {$j \leq 2^{l-i-1}$}
\STATE 			diff[$i+1$][$j$]:=$\frac{1}{2}\frac{\text{diff}[i][2j]-\text{diff}[i][2j+1]}{\text{root}[2j]}$
\STATE 		root = roots($r,t,u,l-1-i$)
\ENDFOR
\ENDFOR
\RETURN diff$[l][0]$
\end{algorithmic}
\end{algorithm}





\begin{algorithm}
\caption{\textsc{DLG}($p,p^\prime,l,x, \epsilon$)}
\label{alg:rational_angle_approx}
\begin{algorithmic}
\STATE angle     := $\frac{1}{2\pi i} \log (x)$
\STATE $u $    := $2^{\epsilon}$
\STATE$t$      :=  $(\text{angle} \cdot u)\% 1$
\STATE $r$      := $|x|$
\RETURN \textsc{DLG\_Rational\_Form}($p,p^\prime,r,t,u,l$)
\end{algorithmic}
\end{algorithm}

%
% u     = pod(2,epsilon)
% t     = mp.fmod(angle,pod(2,-epsilon))
% r     = mpf(mp.fabs(x))




% def DLG_rational_form(p,dp,r,t,u,l):
% 	root       = roots(r,t,u,l)
% 	base_step  = [dp(r)*np.reciprocal(p(r)) for r in root]
% 	derivs     = [base_step]
% 	for i in range(l):
% 		derivs.append([])
% 		for j in range(2**(l-i-1)):
% 			derivs[i+1].append((np.reciprocal(root[2*j])/2)*(derivs[i][2*j] - derivs[i][2*j+1]))
% 		root = roots(r,t,u,l-1-i)
% 	return derivs[l][0]








\section{Theoretical Analysis}
% \newpage
% \clearpage
% \begin{lemma}
% The (relative) condition number operator satisifies the following properties:
% \begin{enumerate}
%   \item $\kappa\{f\} (x) = |x \log'(f(x)) |$
%   \item $\kappa\{f-g\}(x) = |x \frac{ \kappa \{f\}(x)- \kappa \{g\}(x) }{f(x)-g(x)}| $
%   \item $\kappa \left\{\frac{f}{g}\right\} (x)= ||\kappa\{ f\} (x)| - |\kappa\{ g\} (x)||$
%   \item $\kappa \left\{ f \circ g \right\}(x) = \left| |\kappa\{f\} (g(x) )\cdot \kappa\{g\} (x)|\right| $
% \end{enumerate}
% \end{lemma}
% \begin{theorem}
%   The condition number for $\frac{p_l'}{p_l}$ using
% \end{theorem}
\begin{theorem}
Algorithm.~\ref{alg:DLG_rational_form} performs $q \log q$ floating point subtractions, divisions, and multiplications and $q \log q$ applications of $\sin$ and $\cos$, where $q = 2^l$; furthermore, Algorithm.~\ref{alg:DLG_rational_form} performs at most $q \log q$  integer additions, ``multiplications-by-2'', and $ \%2^\epsilon $ (\emph{i.e.,} mod $ 2^\epsilon $) operations.
\end{theorem}
\section{Experimental Results}
\begin{table*}[t]
\caption{Experimental Data for Chebyshev.}
\label{tab:chebyshev}
\vskip 0.15in
\begin{center}
\begin{small}
\begin{sc}
\begin{tabular}{lccccccc}
\toprule
&  &  & mpmath & relative  & relative &  & mpsolve \\
degree  & $l$& $e=-\log(|x|)$& Precision &error $r_d$       & error $r_1$ &runtime& root radius\\
\midrule
 20 & 4 &616 &332 & 0.15&0.09 &0.17 & $[0.00982, 1.0]$ \\
 40 & & & & & & & \\
 80 & & & & & & & \\
 160 & & & & & & & \\
 320 & & & & & & & \\
\bottomrule
\end{tabular}
\end{sc}
\end{small}
\end{center}
\vskip -0.1in
\end{table*}

% \begin{figure}[h]
%   \centering
%   \includegraphics[width=\linewidth]{limit_test.png}
%   \caption{Limit test.}
%   \Description{The procedure for taking the limit to zero.}
% \end{figure}


\section{Conclusion}

%%
%% The acknowledgments section is defined using the "acks" environment
%% (and NOT an unnumbered section). This ensures the proper
%% identification of the section in the article metadata, and the
%% consistent spelling of the heading.
% \begin{acks}
% To Robert, for the bagels and explaining CMYK and color spaces.
% \end{acks}

%%
%% The next two lines define the bibliography style to be used, and
%% the bibliography file.
\bibliographystyle{ACM-Reference-Format}
\bibliography{ref.bib}

%%
%% If your work has an appendix, this is the place to put it.
\appendix



\end{document}
\endinput
%%
%% End of file `sample-sigconf.tex'.



% Freely use this section and cite personal correpsondence with Dr. Pan


% \section{Pan Stuff}
% %------------------------------------------------------------------------------
% \subsection{Root radius and root-squaring}\label{srrapp}
%
% %------------------------------------------------------------------------------.
% The algorithm of Sec. \ref{scnttstB1} and
%  \ref{scnttstB} only ensures
%  approximation of a root radius
% within  factors  of  $d^{1.25}$ and
% $d^{4.75}/m$, respectively (see   Algs.  \ref{algscrincl}$m$ and \ref{algscrcmpr}), versus  a factor of $5^{1/N}$ for, say, $N=3$  ensured by Thm. \ref{thtrn}, and this  is translated into the same error factors for estimated rigidity of the output disc.
% The  difference little affects our estimates for the overall cost of subdivision root-finders (see part (c) of Remark
% \ref{recmprlg}), but not so in some other important applications, e.g., in the following two cases:
% \begin{itemize}
% \item
% A disc $D(c,\rho)$ contains  a root if and only if
% $r_d(c,p)\le \rho$.
% \item
%  The estimates for $r_d(c,p')$  are
%   involved  in  path-lifting Newton's iterations (see  Myong-Hi Kim
%   and Scott Sutherland  \cite{KS94}).
% \end{itemize}
%
% %We also uses a disc $D(c,\alpha r_1(c,p))$ for
% %%\alpha\ge 1$ to initialize  subdivision
% %and Newton's iterations for root-finding.
% This motivates the following problem:
% given
% a range $[\rho_-,\rho_+]$ for  the $m$th smallest root radius, $r_{d-m+1}$, narrow this range.
% We already addressed this problem
%  in Remark \ref{retrhob}, but next we cover alternative  refinement of root radii  approximation, based on the classical technique of recursive root-squaring, which also enables us to relax restriction of Rule 2 of Sec. \ref{ssbdbrf} on softness of exclusion test and to increase isolation of a disc (see, e.g.,  Remark \ref{resmplini}).
%
%  First make an input polynomial $p(x)$ monic by scaling it and/or the variable
%   $x$ and then
%  apply $k$ DLG (that is, Dandelin's aka
% Lobachevsky's or Gr{\"a}ffe's) root-squaring iterations
% (cf. \cite{H59}),
% \begin{equation}\label{eqdnd}
%  p_0(x)=\frac{1}{p_d}p(x),~p_{i+1}(x)=(-1)^ dp_i(\sqrt x)
% p_i(-\sqrt x),~i=0,1,\dots.\ell
% \end{equation}
% for a fixed positive integer $\ell$
% (see Remark \ref{rescdnd} below).
% The $i$th  iteration squares the roots of $p_i(x)$ and
% consequently the root radii from the origin, as well as
% the  isolation of the unit disc $D(0,1)$.  Now approximate
% the ratio  $\rho_+/\rho_-$ for the polynomial $p_{\ell}(x)$ within a factor of $\gamma$ and then readily recover approximation of this ratio for $p_0(x)$ and
% $p(x)$ within a factor of
% $\gamma^{1/2^{\ell}}$.
%
% Given the coefficients of
% $p_i(x)$ , e.g., a deflated factor of $p$,  we can reduce the $i$th root-squaring iteration, that is, the computation of
% the coefficients of  $p_{i+1}(x)$, to polynomial multiplication and perform it in $O(d\log(d))$ arithmetic operations.
% Unless the positive integer $\ell$ is small, the absolute values of the coefficients
% of $p_{\ell}(x)$ vary  dramatically, and realistically one should either stop because of severe problems of numerical stability or apply the stable algorithm by Gregorio Malajovich and Jorge  P. Zubelli \cite{MZ01}, which performs a single root-squaring at  arithmetic cost of order $d^2$.
%
% For black box polynomials $p$, however, we apply
% DLG iterations without computing
% the coefficients, and the algorithm turns out to be quite efficient:
%  for $\ell$ iterations  evaluate
%  $p(x)$ at  $2^{\ell}$ equally spaced points  on a circle and obtain
%  the values of the polynomial $p_{\ell}(x)=\prod(x-x_j^{2^{\ell}})$
%  at these $2^{\ell}$ points.
%
% Furthermore,  evaluate the ratio
% $p'(x)/p(x)=p_0'(x)/p_0(x)$
% at these points by applying
% the recurrence
%  \begin{equation}\label{eqdndrt} \frac{p_{i+1}'(x)}{p_{i+1}(x)}=\frac{1}{2\sqrt x}\Big(\frac{p_{i}'(\sqrt x~)}{p_{i}(\sqrt x~)}-\frac{p_{i}'(-\sqrt x~)}{p_{i}(-\sqrt x~)}\Big),~i=0,1,\dots
% \end{equation}
% Recurrences (\ref{eqdnd}) and (\ref{eqdndrt}) reduce evaluation  of  $p_{\ell}(c)$ to the evaluation of $p(c)$ at  $q=2^{\ell}$  points
% $c^{1/q}$ and  for $c\neq 0$ reduce evaluation  of
%  the ratio
% $p_{\ell}'(x)/p_{\ell}(x)$ at $x=c$ to the evaluation
% of the ratio
% $p'(x)/p(x)$ at the latter $q=2^{\ell}$ points $x=c^{1/q}$.
% We apply  recurrence (\ref{eqdndrt})
% to support application of  Cor. \ref{coqudr} (see Remark \ref{resmplini}).
%
% For $x=0$ recurrence (\ref{eqdndrt}) can be specialized as follows:
%  \begin{equation}\label{eqdndrt0} \frac{p_{{\ell}}'(0)}{p_{\ell}(0)}=\Big(\frac{p_{\ell-1}'(x)}{p_{\ell-1}(x)}\Big)_{x=0}'=\Big(\frac{p'(x)}{p(x)}\Big)_{x=0}^{(\ell)},~\ell=1,2,\dots
% \end{equation}
% Notice an immediate extension:
% \begin{equation}\label{eqdndrth} \frac{p_{{\ell}}^{(h)}(0)}{p_{\ell}(0)}=\prod_{g=1}^h\Big(\frac{p^{(g)}(x)}{p^{(g-1)}(x)}\Big)_{x=0}^{(\ell)},~h=1,2,\dots.
% \end{equation}
% Equations (\ref{eqdndrt0}) and more generally (\ref{eqdndrth})
% enable us to  strengthen
% upper estimates  (\ref{eqrtrdbndsrev1})
% and more generally (\ref{eqrtrdbnds-+i}) for  root-radii  $r_j(0,p)$   at the origin   because $r_j(0,p_{\ell})=r_j(0,p)^{2^{\ell}}$ for $j=1,\dots,d$ (see  (\ref{eqratio0})); we can  approximate the higher order derivatives
% $\Big(\frac{p^{(g)}(x)}{p^{(g-1)}(x)}\Big)^{(\ell)}$
%  at $x=0$ by following Remark \ref{rehighr}.
% Besides listed applications of
% root-squaring, we recall one in Sec. \ref{scorrcch}, for decreasing the softness  of Alg. \ref{algexclrndspr}.
% One can apply root-squaring $p(x)\rightarrow p_{\ell}(x)$ to improve  the error  bound of Cor. \ref{copwrsm} for the approximation of   the power sums  of the roots of  $p(x)$ in the unit disc $D(0,1)$ by Cauchy sums, but the improvement is about as much and at the same additional cost as by increasing the number $q$ of points of evaluation of the ratio $p'/p$.
% % \begin{remark}\label{rescdnd}
%  One can approximate the leading coefficient $p_d$  of a black box polynomial $p(x)$ based on equation (\ref{eqpolyrevlc}). This coefficient is not involved in recurrence (\ref{eqdndrt}),
%  and one can apply  recurrence (\ref{eqdnd}) by using a crude
%  approximation to  $p_d$
%  and if needed can scale polynomials
%  $p_i(x)$ for some $i$.
% % \end{remark}
